\documentclass{beamer}
\usepackage[utf8]{inputenc}
\usepackage[pdf,tmpdir]{graphviz}
\usepackage{graphicx}
\usepackage[dvipsnames]{xcolor}
\usepackage{mathrsfs}
\usepackage{bbold}
\usepackage[export]{adjustbox}
\usepackage{stmaryrd}

\setbeamertemplate{navigation symbols}{}
\usefonttheme[onlymath]{serif}
\graphicspath{{./assets/}}

\definecolor{msg-color}{RGB}{77,0,0}
\definecolor{keyword}{RGB}{8,0,153}

\newcommand{\msgtag}[1]{\texttt{\textcolor{msg-color}{#1}}}
\newcommand{\msgstore}[2]{\msgtag{#1}[\overline{#2}]}
\newcommand{\msgreceive}[2]{\msgtag{#1}(\overline{#2})}
\newcommand{\done}{\texttt{\textcolor{keyword}{done}}}
\newcommand{\free}[1]{\texttt{\textcolor{keyword}{free}}\hspace{0.25em}#1}
\newcommand{\fail}[1]{\texttt{\textcolor{keyword}{fail}}\hspace{0.25em}#1}
\newcommand{\parbar}{\kern3pt\rule[-3pt]{0.8pt}{0.9\baselineskip}\kern3pt}

\title{Mailbox Types for Unordered Interactions\footnote{\tiny{Ugo de'Liguoro, Luca Padovani (2018). https://arxiv.org/abs/1801.04167}}}
\author{Presentación por Jonathan Bekenstein}
\institute{Materia optativa sobre Tipos Comportamentales y Contratos}
\date{2024}

\begin{document}

\frame{\titlepage}

\begin{frame}{Introducción}
    Modelar interacciones \textbf{asincrónicas} sobre diferentes topologías de procesos concurrentes. La comunicación es \textbf{multi-party} y sucede mediante el uso de un \textbf{mailbox}, en donde:
    \begin{itemize}
        \item procesos pueden escribir mensajes identificados por un \emph{tag} con argumentos opcionales,
        \item procesos pueden consumir (leer) los mensajes en un orden arbitrario (\emph{out-of-order} o \emph{selective processing}).
    \end{itemize}
    \center
    \vspace{-1em}
    \digraph[scale=0.5]{intro}{
        graph [rankdir=LR, pad=0.1]
        node [shape=circle, fontname="Latin Modern Mono"]
        mailbox [shape=box]
        P1 -> mailbox
        P2 -> mailbox
        P3 -> mailbox
        mailbox -> Q
    }
\end{frame}

\begin{frame}{Mailbox calculus}{Sintaxis: gramática}
    \begin{figure}[H]
        \centering
        \includegraphics[width=0.8\textwidth]{syntax}
    \end{figure}

    \footnotesize{$\overline{u}$ denota la secuencia $u_1, \dots, u_n$}
\end{frame}

\begin{frame}{Mailbox calculus}{Sintaxis: mensajes}
    \begin{block}{Enviar mensajes}
        $u!\msgstore{m}{v}$
        \\ Guarda un mensaje en el mailbox $u$ identificado con el tag $\msgtag{m}$ y argumentos $\overline{v}$.
    \end{block}

    \begin{block}{Recibir mensajes}
        $u?\msgreceive{m}{x}.P$ \\ Consume selectivamente el mensaje con tag $\msgtag{m}$ del mailbox $u$ y continúa con $P$ reemplazando $\overline{x}$ por los argumentos del mensaje.
    \end{block}
\end{frame}

\begin{frame}{Mailbox calculus}{Sintaxis: procesos}
    \begin{block}{Invocación}
        $X[\overline{u}]$ representa la invocación de un proceso llamado $X$ con parámetros $\overline{u}$. Asumimos que existe una definición global de procesos de la forma $X(\overline{x}) \triangleq P$.
    \end{block}

    \begin{block}{Paralelo y restricción}
        $P \parbar Q$ denota la composición paralela de procesos, y $(\nu a) P$ representa un mailbox $a$ restringido al scope de $P$.
    \end{block}

    \begin{block}{Terminación}
        Un proceso $\done$ representa un proceso terminado y que no realiza ninguna otra acción.
    \end{block}
\end{frame}

\begin{frame}{Mailbox calculus}{Sintaxis: guardas}
    \begin{block}{Guardas}
        Las guardas $G$ y la composición de guardas $G+H$ nos permite modelar distintas ``ramas'' de ejecución en función del mensaje consumido del mailbox. Luego se usa exclusivamente la continuación de la guarda que consumió el mensaje.
    \end{block}

    \begin{block}{Errores}
        $\fail{u}$ permite modelar un runtime error al recibir un mensaje inesperado.
    \end{block}

    \begin{block}{Eliminar mailbox}
        $\free{u}.P$ permite eliminar el mailbox $u$ si ya no se va a utilizar y continuar la ejecución con $P$.
    \end{block}
\end{frame}

\begin{frame}{Mailbox calculus}{Semántica operacional}
    \begin{block}{Reglas de reducción}
        \begin{figure}[H]
            \centering
            \includegraphics[width=0.9\textwidth]{reduction-rules}
        \end{figure}
    \end{block}

    \begin{block}{Relación de congruencia estructural}
        \begin{figure}[H]
            \centering
            \includegraphics[width=0.9\textwidth]{structural-congruence}
        \end{figure}
    \end{block}
\end{frame}

\begin{frame}{Ejemplo 1: Lock}
    \begin{figure}[H]
        \centering
        \includegraphics[width=\textwidth]{example1-lock}
    \end{figure}
    \vspace{-1em}
    \begin{figure}[H]
        \centering
        \includegraphics[width=\textwidth]{example1-user}
    \end{figure}
    \vspace{-1em}
    \begin{figure}[H]
        \centering
        \includegraphics[width=\textwidth]{example1-usage}
    \end{figure}

    \begin{block}{Observaciones}
        \begin{itemize}
            \item \texttt{FreeLock} consume de manera no determinística los mensajes \msgtag{acquire}.
            \item \texttt{User} utiliza la referencia $l$ para enviar el mensaje \msgtag{release} ya que es ésta la referencia al mailbox que tiene la capabilidad de procesar este mensaje.
        \end{itemize}
    \end{block}
\end{frame}

\begin{frame}{Mailbox calculus}{Caracterizaciones operacionales}
    \begin{block}{Contextos de procesos}
        \vspace{-0.5em}
        \begin{figure}[H]
            \includegraphics[width=0.7\textwidth,left]{process-context}
        \end{figure}
        \vspace{-1em}
        Los contextos de procesos buscan identificar un ``unguarded hole'', es decir un agujero que no tiene prefijada una acción sobre un mailbox.
    \end{block}
    \begin{block}{Def 4: Mailbox conformant}
        $P$ es \emph{mailbox conformant} si $P \not\rightarrow^\ast \mathscr{C}[\fail{a}]$ para todo $\mathscr{C}$ y $a$.
        \\
        En el ejemplo del lock, ser \emph{mailbox conformant} significa nunca liberar el lock antes de adquirirlo.
    \end{block}
\end{frame}

\begin{frame}{Mailbox calculus}{Caracterizaciones operacionales}
    \begin{block}{Def 5: Deadlock free}
        $P$ es \emph{deadlock free} si $P \rightarrow^\ast Q \not\rightarrow$ implica $Q \equiv \done$.
        \\
        Un proceso se considera \emph{deadlock free} si al terminar tenemos que (1) no hay subprocesos esperando un mensaje que nunca se va a producir y (2) todos los mailbox están vacíos.
    \end{block}
    \begin{block}{Def 7: Fairly terminating}
        $P$ es \emph{fairly terminating} si $P \rightarrow^\ast Q$ implica que $Q \rightarrow^\ast \done$.
        \\
        Es una propiedad más fuerte que deadlock freedom. Si un proceso es \emph{fairly terminating} entonces se garantiza \emph{junk freedom} (no quedan mensajes sin consumir en ningún mailbox).
    \end{block}
\end{frame}

\begin{frame}{Ejemplo 2: Future variable}
    \begin{figure}[H]
        \includegraphics[width=\textwidth]{example2-future}
    \end{figure}
    \begin{figure}[H]
        \includegraphics[width=\textwidth]{example2-deadlock}
    \end{figure}

    Es legal hacer un \msgtag{get} de una variable futura antes de que sea resuelta con \msgtag{resolve}. Este proceso es \emph{mailbox conformant} pero está en deadlock. Para evitar que este proceso esté bien tipado se usan \emph{dependency graphs}.
\end{frame}

\begin{frame}{Mailbox type system}{Sintaxis}
    \begin{figure}[H]
        \includegraphics[width=0.9\textwidth]{type-syntax}
    \end{figure}
\end{frame}

\begin{frame}{Mailbox type system}{Patrones}
    Los patrones son \emph{expresiones regulares conmutativas} que describen las configuraciones válidas de los mensajes dentro de un mailbox.
    \vspace{1em}
    \begin{itemize}
        \item $\mathbb{0}$: \emph{unreliable mailbox} que recibió un mensaje inesperado.
        \item $\msgtag{A} + \msgtag{B}$: contiene un mensaje $\msgtag{A}$ o un mensaje $\msgtag{B}$ pero no ambos.
        \item $\msgtag{A} + \mathbb{1}$: contiene un mensaje $\msgtag{A}$ o está vacío.
        \item $\msgtag{A} \cdot \msgtag{B}$: contiene un mensaje $\msgtag{A}$ y un mensaje $\msgtag{B}$.
        \item $\msgtag{A}^\ast$: contiene un cantidad arbitraria (incluso $0$) de mensajes $\msgtag{A}$.
    \end{itemize}
\end{frame}

\begin{frame}{Mailbox type system}{Capabilities}
    Un \emph{mailbox type} consiste en un \emph{capability} ($?$ o $!$) junto a un patrón. Un proceso debe cumplir ciertas obligaciones y tiene ciertas garantías descriptas por el mailbox type asociado al mailbox que usa.
    \vspace{1em}
    \begin{itemize}
        \item $!\msgtag{A}$: el proceso \textbf{debe} escribir un mensaje $\msgtag{A}$ en el mailbox.
        \item $?\msgtag{A}$: el proceso tiene \textbf{garantizado} recibir un mensaje $\msgtag{A}$.
    \end{itemize}
\end{frame}

\begin{frame}{Mailbox type system}{Capabilities: más ejemplos}
    \vspace{1em}
    \begin{itemize}
        \item $!(\msgtag{A} + \mathbb{1})$: el proceso \textbf{puede} escribir un mensaje $\msgtag{A}$ en el mailbox, pero no está obligado a hacerlo.
        \item $!(\msgtag{A} + \msgtag{B})$: el proceso \textbf{debe} escribir un mensaje $\msgtag{A}$ o $\msgtag{B}$, pero puede \textbf{elegir} cuál.
        \item $?(\msgtag{A} + \msgtag{B})$: el proceso \textbf{debe} estar preparado para recibir tanto un mensaje $\msgtag{A}$ como $\msgtag{B}$.
        \item $?(\msgtag{A} \cdot \msgtag{B})$: el proceso tiene \textbf{garantizado} recibir un mensaje $\msgtag{A}$ y otro $\msgtag{B}$, y puede elegir en qué orden recibirlos.
        \item $!(\msgtag{A} \cdot \msgtag{B})$: el proceso \textbf{debe} escribir un mensaje $\msgtag{A}$ y otro $\msgtag{B}$.
        \item $!\msgtag{A}^\ast$: el proceso \textbf{elige} cuántos mensajes $\msgtag{A}$ escribir.
        \item $?\msgtag{A}^\ast$: el proceso \textbf{debe} estar preparado para recibir una cantidad arbitraria de mensajes $\msgtag{A}$.
    \end{itemize}
\end{frame}

\begin{frame}{Mailbox type system}{Semántica de patrones}
    La semántica de los patrones se define como conjuntos de multiconjuntos de átomos: $\msgstore{m}{\tau}$.

    \begin{figure}[H]
        \includegraphics[width=\textwidth]{subpattern}
    \end{figure}

    Dada una relación preorder $\mathscr{R}$ sobre los tipos básicos, escribimos $E \sqsubseteq_\mathscr{R} F$ para decir que E es un subpatrón de F si $\langle \msgtag{m}_i[\overline{\tau}_i] \rangle_{i \in I} \in \llbracket E \rrbracket$ implica $\langle \msgtag{m}_i[\overline{\sigma}_i] \rangle_{i \in I} \in \llbracket F \rrbracket$ y además $\overline{\tau}_i \hspace{0.25em} \mathscr{R} \hspace{0.25em} \overline{\sigma}_i$ para todo $i \in I$.
    \vspace{1em}

    Escribimos $\simeq_\mathscr{R}$ para denotar $\sqsubseteq_\mathscr{R} \cap \sqsupseteq_\mathscr{R}$.
    \vspace{1em}

    Notemos que $\sqsubseteq_\mathscr{R}$ es covariante respecto a $\mathscr{R}$,
    \\ pues $\overline{\tau} \hspace{0.25em} \mathscr{R} \hspace{0.25em} \overline{\sigma}$ implica $\msgstore{m}{\tau} \sqsubseteq_\mathscr{R} \msgstore{m}{\sigma}$.
\end{frame}

\begin{frame}{Mailbox type system}{Subtipado}
    Decimos que $\mathscr{R}$ es una \emph{relación de subtipado} si $\tau \hspace{0.25em} \mathscr{R} \hspace{0.25em} \sigma$ implica
    \begin{enumerate}
        \item $\tau = \hspace{0.3em} ?E$ y $\sigma = \hspace{0.3em} ?F$ y $E \sqsubseteq_\mathscr{R} F$, o bien
        \item $\tau = \hspace{0.3em} !E$ y $\sigma = \hspace{0.3em} !F$ y $F \sqsubseteq_\mathscr{R} E$
    \end{enumerate}
    \vspace{1em}

    Escribimos $\leqslant$ para denotar la mayor relación de subtipado y decimos que $\tau$ es un subtipo de $\sigma$ si $\tau \leqslant \sigma$.
    \vspace{1em}

    Escribimos $\lessgtr$ para $\leqslant \cap \geqslant$, $\sqsubseteq$ para $\sqsubseteq_{\leqslant}$ y $\simeq$ para $\simeq_{\leqslant}$.
    \vspace{1em}

    Las 2 reglas se corresponden directamente con las reglas covariantes y contravariantes usuales para canales con capacidades de entrada y salida.
\end{frame}

\begin{frame}{Mailbox type system}{Subtipado: ejemplos}
    \begin{itemize}
        \item $?\msgtag{A} \leqslant \hspace{0.3em} ?(\msgtag{A} + \msgtag{B})$: un mailbox de tipo $?\msgtag{A}$ ofrece garantías más fuertes que otro de tipo $?(\msgtag{A} + \msgtag{B})$. Si un proceso sabe usar un mailbox donde pueden haber mensajes $\msgtag{A}$ o $\msgtag{B}$, también sabe usar un mailbox donde solo hay mensajes $\msgtag{A}$.
        \item $!(\msgtag{A} + \msgtag{B}) \leqslant \hspace{0.3em} !\msgtag{A}$: un mailbox de tipo $!(\msgtag{A} + \msgtag{B})$ es más permisivo que otro de tipo $!\msgtag{A}$. Si un proceso necesita un mailbox para escribir un mensaje $\msgtag{A}$, también le sirve un mailbox donde se puede escribir mensajes $\msgtag{A}$ o $\msgtag{B}$.
    \end{itemize}
\end{frame}


\begin{frame}{Mailbox type system}{Relaciones de subtipado especiales}
    Algunos mailbox types tienen patrones que están en cierta relación particular con las constantes $\mathbb{0}$ (unreliable mailbox) y $\mathbb{1}$ (empty mailbox). Estos tipos tienen la siguiente clasificación:
    \vspace{1em}

    \begin{itemize}
        \item \textbf{relevant}: si $\tau \not\leqslant \hspace{0.3em} !\mathbb{1}$ (caso contrario \emph{irrelevant})
        \\ Un mailbox \emph{relevant} debe usarse, mientras que uno \emph{irrelevant} puede descartarse. Todos los mailbox con input capability son \emph{relevantes}.
        \item \textbf{reliable}: si $\tau \not\leqslant \hspace{0.3em} ?\mathbb{0}$ (caso contrario \emph{unreliable})
        \\ Un mailbox \emph{reliable} no recibió mensajes inesperados. Todos los mailbox con output capability son \emph{reliable}.
        \item \textbf{usable}: si $!\mathbb{0} \not\leqslant \hspace{0.2em} \tau$ (caso contrario \emph{unusable})
        \\ Un mailbox \emph{usable} puede ser usado. Todos los mailbox con input capability son \emph{usable}.
    \end{itemize}
\end{frame}

\begin{frame}{Mailbox type system}{Ejemplo 11: lock type}
    \begin{figure}[H]
        \centering
        \includegraphics[width=\textwidth]{example1-lock}
    \end{figure}

    El mailbox usado por FreeLock tendrá diferentes tipos dependiendo del estado interno del lock y desde qué óptica lo miramos.
    \vspace{1em}

    \begin{itemize}
        \item Desde FreeLock: $?\msgtag{acquire}[!\msgtag{reply}[!\msgtag{release}]]^\ast$
        \item Desde BusyLock: $?(\msgtag{release} \cdot \msgtag{acquire}[!\msgtag{reply}[!\msgtag{release}]]^\ast)$
        \item Desde User hacia FreeLock: $!\msgtag{acquire}[!\msgtag{reply}[!\msgtag{release}]]^\ast$
        \item Desde Owner hacia BusyLock: $!\msgtag{release}$

    \end{itemize}
\end{frame}

\begin{frame}{Mailbox type system}{Grafo de dependencias: sintaxis}
    \begin{figure}[H]
        \includegraphics[width=\textwidth]{dependency-graph-syntax}
    \end{figure}

    El grafo de dependencias es un multigrafo no dirigido donde los vértices del grafo son los nombres de los mailbox y el \textbf{objetivo es trackear las dependencias entre mailboxes}. Intuitivamente hay una dependencia entre $u$ y $v$ si:
    \vspace{1em}

    \begin{itemize}
        \item $v$ es un argumento de algún mensaje del mailbox $u$,
        \item o bien $v$ aparece en la continuación de un proceso esperando un mensaje en $u$.
    \end{itemize}
\end{frame}

\begin{frame}{Mailbox type system}{Grafo de dependencias: LTS}
    \begin{figure}[H]
        \includegraphics[width=\textwidth]{dependency-graph-lts}
    \end{figure}

    La semántica del grafo de dependencias está dada por un LTS donde el label $u - v$ representa un camino que conecta $u$ con $v$. La relación $\varphi \xrightarrow{u-v} \varphi'$ describe que $u$ está conectado con $v$ en $\varphi$, y $\varphi'$ describe el grafo residual luego de eliminar las aristas usadas para conectar $u$ con $v$.
\end{frame}

\begin{frame}{Mailbox type system}{Grafo de dependencias: propiedades}
    \begin{block}{Def 12: graph acyclicity and entailment}
        Sea $\text{dep}(\varphi) \stackrel{\text{def}}{=} \{ (u,v) \mid \exists \varphi' : \varphi \xrightarrow{u-v} \varphi' \}$ la relación de dependencias generada por $\varphi$.
        \begin{itemize}
            \item Decimos que $\varphi$ es \emph{acíclico} si $\text{dep}(\varphi)$ es irreflexiva.
            \item Decimos que $\varphi$ \emph{entails} (implica) $\psi$, escrito $\varphi \Rightarrow \psi$, si $\text{dep}(\psi) \subseteq \text{dep}(\varphi)$.
        \end{itemize}
    \end{block}
\end{frame}

\begin{frame}{Mailbox type system}{Grafo de dependencias: ejemplo}
    \begin{figure}[H]
        \includegraphics[width=\textwidth]{dependency-graph-example}
    \end{figure}

    \begin{itemize}
        \item El mailbox $c$ es el argumento del mensaje $\msgtag{get}$ guardado en el mailbox $f$.
        \item El mailbox $f$ aparece en la continuación luego de leer del mailbox $c$.
    \end{itemize}

    \vspace{-1em}
    \begin{figure}
        \centering
        \digraph[scale=0.5]{dependencygraph}{
            graph [rankdir=LR, pad=0.1]
            node [shape=circle, fontname="Latin Modern Mono"]
            edge [dir=none]
            f -> c
            f -> c
        }
    \end{figure}
    \vspace{-1em}

    Claramente el grafo de dependencias \textbf{tiene un ciclo}. Vamos a utilizar estos grafos en las reglas de tipado para evitar deadlocks.
\end{frame}

\begin{frame}{Mailbox type system}{Reglas de tipado}
    Type environments
\end{frame}

\begin{frame}{Mailbox type system}{Propiedades}
    \begin{block}{Teorema 23}
    \end{block}
    \begin{block}{Teorema 24}
    \end{block}
    \begin{block}{Teorema 25}
    \end{block}
\end{frame}

\begin{frame}{Encoding binary sessions}
\end{frame}

\end{document}
